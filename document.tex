\documentclass[a4paper,12pt]{article}
%\documentclass[UTF8]{article}
%\documentclass[compress]{elsarticle}

%%%%%%%%%%%%%%%%%%%%%%%%%%%%%%%%%%%%%%%%%%%%%%%%%%%%%%%%%%%%%%%%%%%%%%%%%%%%%%%%%%%%%%%%%%%%%%%%%%%%%%%%%%%%%%%%%%%%%%%%%%%%%%%%%%%%%%%%%%%%%%%%%%%%%%%%%%%%

\usepackage{times}
\usepackage{attrib}
\usepackage[hyperfootnotes=false]{hyperref}
\usepackage{bookmark}
\usepackage{footnote}
\usepackage[onehalfspacing]{setspace}
\usepackage{lscape}
\usepackage{geometry}
\usepackage{amsmath, amssymb, graphicx}
\usepackage{booktabs, topcapt}
\usepackage{threeparttable}
\usepackage{array}
\usepackage{paralist}
\usepackage{tikz}
\usepackage{verbatim}
\usepackage{multirow}
\usepackage{rotating}
\usepackage{lscape}
\usepackage{color}
\usepackage{array}
\usepackage{colortbl}
\usepackage{bm}
\usepackage{ctex}
\usepackage{bbm}
\usepackage{indentfirst}
\usepackage{graphicx}
\usepackage{float}
\usepackage{caption}
%\usepackage[numbers,sort&compress]{natbib}

\usepackage{cite}
\newcommand{\upcite}[1]{\textsuperscript{\textsuperscript{\cite{#1}}}}
%\newcommand{\upcite}[1]{$^{\mbox{\scriptsize \cite{#1}}}$}
\makeatletter
\def\@cite#1#2{\textsuperscript{[{#1\if@tempswa , #2\fi}]}}
\makeatother

\newcommand{\sym}[1]{#1}
%\usepackage[backend=biber, isbn=false, doi=false]{biblatex-chicago}
\usepackage[authoryear]{natbib}
%\usepackage[perpage,symbol*]{footmisc}	%脚注变圆圈
%\usepackage{pifont}		%脚注变圆圈


\setcounter{MaxMatrixCols}{10}
\bookmarksetup{numbered}
\usetikzlibrary{patterns}
\geometry{left=1.0in,right=1.0in,top=1.5in,bottom=1.5in}
\setlength{\parindent}{2em} %首行缩进两字符
%\DefineFNsymbols{circled}{{\ding{192}}{\ding{193}}{\ding{194}}
%	{\ding{195}}{\ding{196}}{\ding{197}}{\ding{198}}{\ding{199}}{\ding{200}}{\ding{201}}}
%\setfnsymbol{circled}		%脚注变圆圈

%% 提示 %%
%| 字体大小        | 号码     |
%| ------------- | ------- |
%| \tiny         | 5pt     |
%| \scriptsize   | 7pt     |
%| \footnotesize | 8pt     |
%| \small        | 9pt     |
%| \normalsize   | 10pt    |
%| \large        | 12pt    |
%| \Large        | 14.4pt  |
%| \LARGE        | 17.28pt |
%| \huge         | 20.74pt |
%| \Huge         | 24.88pt |



\begin{document}
	% 这是一个CTEX的utf-8编码例子,{\kaishu 这里是楷体显示},{\songti 这里是宋体显示},{\heiti 这里是黑体显示},{\fangsong 这里是仿宋显示}。

\title{\centering 对外承包工程促进了装备制造业出口吗? 
	 \thanks{对外经济贸易大学中央高校基本科研业务费专项资金资助(19QD27)} }
\author{\addtocounter{footnote}{1} {\normalsize 李小帆 \ \ \ 孟克} 
	\thanks{李小帆(1990-),男,四川内江人,对外经济贸易大学中国WTO研究院助理研究员;孟克(1997-),男,天津人,对外经济贸易大学中国WTO研究院博士生,本文通讯作者。} \\}
\date{\today}
\maketitle

\begin{abstract}
	中国对外承包工程的迅速发展既是中国工程建设能力的一张名片,也是帮助中国装备制造业打破欧美垄断,开拓国际市场的重要推动力。本文以2009年中石化承包中国在科威特的首个石油钻井项目为例展开案例分析。在当时科威特要求统一配备欧美装备的背景下,中石化努力说服对方以“试用”的方式引进中国装备,并通过积极推介和标准协调,帮助中方企业进入科威特市场。数据分析表明,2009年之后中国制造的钻井装备及配套产品在科威特的市场份额激增。本文进一步通过双重差分法验证了中国企业对外承包工程对国产工程装备及配套产品出口的促进作用,并且发现在“一带一路”倡议提出之后对沿线国家的促进作用增强。最后,相对于对外直接投资,对外承包工程的出口促进效应更强。 \\
	\medskip
	\textbf{Keywords:} 对外承包工程;机械装备出口; 双重差分法 \\
	\textbf{JEL Codes: }F752.7
\end{abstract}

\bigskip

\newpage	


\section{引言}
习近平总书记多次强调,装备制造业是制造业的脊梁。不断开拓和占领世界市场是中国装备制造业做大做强的重要途径。然而,作为后起的工业化强国,中国装备制造业打入国际市场的过程充满挑战。大型机械装备具有典型的单价高、购买次数和购买数量有限的特点。因此,在21世纪初欧美发达国家占据绝大部分国际市场的情况下,如果没有外力的推动,国外企业很难会尝试使用中国制造的机械装备,相应地,中国装备制造业也就很难进入国际市场。中国企业参与对外承包工程则为国外企业了解中国机械装备提供了良好契机,并有望成为中国机械装备走出国门的重要突破口。

	\vspace{0.5em}  %段前0.5行

对外直接投资和对外承包工程是中国企业“走出去”的两种形式,但学界对后者的关注远远不够。实际上,进入21世纪以来,中国企业积极参与对外承包工程,在基建设施、能源勘探和建筑工程等领域取得了辉煌成绩。截至2019年底,累计签订合同金额达2.58万亿,累计完成营业额1.76万亿\footnote{①	数据源自《中国对外承包工程国别(地区)市场报告2019~2020》。},两者较2005年均增长接近70倍。尤其是在共建“一带一路”倡议引领下,对外承包工程更是进入加速发展阶段,为中国装备制造业的出口和发展带来新的机遇。正如习近平总书记所指出的,建设“一带一路”,装备制造业大有可为之机。因此,本文拟研究中国企业参与对外承包工程能否成为中国装备制造业出口的重要推力。

	\vspace{0.5em}  %段前0.5行
	
从具体案例来看,2009年中石化在科威特获得中国在该国第一个石油钻井项目。中石化努力说服科威特合作方尝试使用中国装备,同时积极向科威特其他厂商推介中国机械装备,并帮助国内生产企业对接国际标准。数据表明,在中石化的积极推动下,中国在科威特石油钻井相关装备产品的市场份额在2009年之后激增,并保持稳定的增长态势,中国制造的机械装备逐渐占据科威特市场。科威特石油钻井项目承包案例充分体现了通过对外承包工程促进中国装备制造业出口的现实可能性。深入分析科威特石油钻井项目承包案例,可以窥探对外承包工程促进中国装备制造业出口的路径。在此基础上,本文进一步利用2005~2019年数据,就对外承包工程对工程类机械装备及配套产品出口的影响进行实证分析。

	\vspace{0.5em}  %段前0.5行
	
本文的研究结果表明,对外承包工程使中国工程产品出口增加近16\%,以累计项目合同金额来衡量对外承包工程时,其出口效应更大;“一带一路”倡议之后,中国企业承包沿线国家工程项目对中国向沿线国家装备出口的促进作用进一步增强。这是因为,在“一带一路”倡议提出之后,中国在沿线国家承包的工程项目更倾向于采用中国标准和中国管理服务,所以在沿线国家的工程承包对出口的影响更大\footnote{例如,中国路桥工程有限责任公司承建的肯尼亚蒙内铁路,全线采用中国标准、中国技术与管理,并由中国企业负责后续运营,这更加有利于项目采用中国制造的装备。正因如此,项目共从国内13家装备供应商采购了装备1450台套,带动中国出口7.4亿美元。}。并且,本文发现对于初始期中国机械装备市场份额较小的国家,对外承包工程对出口的促进作用更加明显。此外,对外承包工程的出口促进效应显著大于对外直接投资\footnote{需要指出的是,本文研究对外直接投资对于出口的影响,是指中国企业在工程类领域的对外直接投资能够带动中国相关装备的出口,这与研究企业的直接投资行为对自身出口影响的研究不同。}。
下文内容安排如下:第二部分为文献综述,第三部分为本文研究背景和典型案例分析,第四部分为实证研究的变量、数据说明以及实证模型设定,第五部分为实证结果。第六部分为结论与政策建议。



\section{文献综述}

现有文献主要从影响因素和作用两方面分析了对外承包工程的相关问题。首先是关于对外承包工程影响因素的研究。曾剑宇等(2017)和何凡等(2018)聚焦“一带一路”沿线国家,分别考察了国家风险和双边关系对中国对外承包工程的影响\upcite{ref1} \upcite{ref2}。蒋骄亮、何凡和曾剑宇(2017)则从外部经济环境的角度进行分析,指出汇率上升及汇率波动均会抑制企业对外承包工程\upcite{ref3}。吕荣杰、张冰冰和张义明(2018)基于VAR模型发现双边贸易能够促进对外承包工程,其中货物出口的影响效应最为明显\upcite{ref4}。

	\vspace{0.5em}  %段前0.5行
	
其次是关于对外承包工程作用的研究。从已有的文献来看,对外承包工程对东道国及母国均存在积极影响。对东道国而言,曾剑宇、何凡和蒋骄亮(2017)指出对外承包工程能够促进东道国产业结构升级\upcite{ref5},李者聪(2021)基于“一带一路”非洲沿线国家发现对外承包工程能够促进东道国的经济增长\upcite{ref6},徐俊和李金叶(2020)验证了中国对外承包工程对东道国基础设施质量的促进作用\upcite{ref7}。对于母国而言,杨忻、刘芳和张国清(2005)验证了对外承包工程对本国国民经济的拉动作用\upcite{ref8},蔡阔等(2013)指出中国对外承包工程能够显著促进对外直接投资\upcite{ref9}。部分文献也关注了对外承包工程的贸易效应。覃伟芳和陈红蕾(2018)利用2002-2006年海关数据,发现中国对外承包工程对中国工业企业出口存在促进作用\upcite{ref10}。喻春娇(2021)发现中国在“一带一路”沿线国家承包工程能够促进中国钢铁产品出口\upcite{ref11}。洪俊杰与詹迁羽(2021)以中国对“一带一路”沿线国家海外工程承包额衡量中国与沿线国家的设施联通程度,分析了中国与沿线国家设施联通程度对中国出口的影响\upcite{ref12}。

	\vspace{0.5em}  %段前0.5行

相较于现有文献,本文的边际贡献体现在以下三个方面。第一,从研究视角看,本文重点关注了对外承包工程对中国装备制造业出口的促进作用。装备制造业是制造业最核心的部门。本文指出,由于装备制造业具有典型的单价高、购买次数和购买数量有限的特点,在欧美高度垄断的背景下,对外承包工程能够有效帮助中国装备制造业进入国际市场。第二,从研究方法看,本文通过具体案例分析提供了对外承包工程促进装备制造业出口的证据,明确了中间渠道,并进一步通过双重差分方法保证了估计的可靠性。第三,本文实证分析的样本选择更具代表性。由于“一带一路”沿线国家主要是发展中国家,所以仅包含沿线国家的实证分析,样本代表性难以保证。区别于已有文献,本文研究的样本不仅包含“一带一路”沿线国家,也包含其他非沿线国家。此外,考虑到中国对外承包工程从2005年之后才开始迅速发展,本文选取了2005~2019年的样本进行分析,这保证了文章结论的代表性。


\section{研究背景和典型案例分析}
\subsection{对外承包工程和装备制造业出口发展情况}
根据商务部“走出去”服务平台的定义,对外承包工程是指中国企业在境外建设工程项目的活动。 “走出去”战略于2000年3月在全国人大九届三次会议被正式提出,旨在使中国企业通过对外承包工程、对外直接投资等形式参与国际竞争与合作。在“走出去”战略的影响下,中国企业顺应经济全球化浪潮,积极投身对外承包工程,在基础设施建设、资源与能源开发和石油勘探等领域逐步开拓国际市场。

	\vspace{0.5em}  %段前0.5行
	
表\ref{table1}展示了自2005年以来,中国企业对外工程项目承包的情况。表\ref{table1}第一列显示了中国企业对外承包项目累积合同数量的增长趋势。如表\ref{table1}所示,2005年中国企业仅有18个对外工程项目合同,但经过持续不断增长,2019年中国企业签署了1364个对外工程项目合同。第二列反映了中国签署的对外工程项目合同累计金额。与合同数量一致,中国签署对外承包工程项目合同累积金额从2005年的81.2亿美元迅速增长至2019年的6772.5亿美元。此外,中国对外工程承包的地域范围也在不断拓展。如表\ref{table1}第三列所示,2005年中国仅在15个国家承包了工程项目,但是到2013年已经在超过100个国家承包工程项目,并且在后续年份还在缓慢增加。


\begin{table} [ht] \footnotesize
	\centering
	\captionsetup{labelformat=default,labelsep=period}
	\caption{中国企业对外承包工程情况}\label{table1}
	\begin{tabular}{lcccc}
		\toprule
			\vspace{0.2em}
	 &&  累积合同个数(单位:个)     & 累积合同金额(单位:亿美元) & 累积涉及国家个数(单位:个) \\
		\cline{1-5} 
		
	   &2005&	18&	81.2&	15 \\
	   &2006&42	&259.5&	24 \\
	   &2007&	79&	482.6&	37 \\
	   &2008&	125&	740.1&	48 \\ 
	   &2009&	188&	1080.3&	63 \\
	   &2010&	267&	1481.2&	74 \\
	   &2011&	361&	1946.6&	83 \\
	   &2012&	449&	2447.7&	97 \\ 
	   &2013&	548&	2934.4&	105 \\ 
	   &2014&	666&	3546.0&	112 \\
	   &2015&	792&	4180.4&	114 \\
	   &2016&	943&	4873.7&	116 \\
	   &2017&	1088&	5570.4&	121 \\
	   &2018&	1223&	6190.0&	128 \\
	   &2019&	1364&	6772.5&	130 \\		
		\bottomrule
	\end{tabular}
	\begin{minipage}{14cm}%
		\scriptsize\vspace{0.5em}
		注:数据来自美国企业研究所(American Enterprise Institute)2020年发布的《中国全球投资跟踪(China Global Investment Tracker)》。此处所指对外工程项目包括“能源开发”“交通设施”“建筑”和“矿产”四类。
	\end{minipage}%
\end{table}

	\vspace{0.5em}  %段前0.5行

从参与者的角度看,大型国有企业是对外承包工程绝对的主力军。如表\ref{table2}所示,在中国参与对外承包工程的所有企业中,国有企业占比90\%,由国有企业签订的项目合同数量占比更是高达98.18\%。相比之下,对外直接投资中国有企业占所有参与者的比重约为61\%,仅为对外承包工程国有企业比重的一半左右。对外直接投资中国有企业签订的项目合同数量占比也比对外承包工程中国有企业合同数量占比少25\%左右。

	\vspace{0.5em}  %段前0.5行


在对外工程项目承包蓬勃发展的同时,中国工程类机械装备及配套产品的出口也在逐年上升。如表\ref{table3}第一列所示,本文关注的工程类装备及相关产品的出口额占比常年维持在45\%以上,是中国出口的重要组成部分。从增长趋势看,如表3第二列所示,中国工程类装备和产品的出口保持了良好的增长态势,并在2014年达到顶峰。在2014年之后,虽然出口略有回落,但总体依然保持在高位。最后从世界市场份额看,如表\ref{table2}第三列所示,中国工程类装备和产品出口在世界市场的份额在2005~2020年间从12\%左右提高到17\%左右,涨幅接近40\%。

	\vspace{0.5em}  %段前0.5行

\begin{table} [ht] \footnotesize
	\newcommand{\tabincell}[2]{\begin{tabular}{@{}#1@{}}#2\end{tabular}}  %导言区
	\centering
	\captionsetup{labelformat=default,labelsep=period}
	\caption{中国对外承包工程的国有企业占比}\label{table2}
	\begin{tabular}{lccccc}
		\toprule
		%\vspace{0.2em}
		&  \multicolumn{2}{c}{对外承包工程}       &&   \multicolumn{2}{c}{对外直接投资} \\
		\cline{2-3} \cline{5-6}
		部门分类 & \tabincell{c}{国有企业参与 \\ 数量占比} & \tabincell{c}{国有企业合 \\ 同数量占比} && \tabincell{c}{国有企业参 \\ 与数量占比} & \tabincell{c}{国有企业合 \\ 同数量占比} \\ 
		能源开发类 & 88.89 & 97.14 && 62.61 & 84.06 \\ 
		交通设施类 & 91.43 & 99.41 && 55.77 & 63.94 \\ 
		建筑类 & 90.24 & 98.39 && 66.67 & 70.18 \\ 
		矿产类 & 94.74 & 97.44 && 63.46 & 76.67 \\ 
		合计 & 90.52 & 98.18 & &61.24 & 76.72 \\ 
		\bottomrule
	\end{tabular}
\end{table}

\vspace{0.5em}  %段前0.5行


经过艰难开拓和激烈竞争,中国工程机械及配套产品凭借自身的价格优势以及过硬的质量,逐渐获得国外认可,在世界市场的份额大幅提升。本文将研究,中国企业参与对外承包工程能否助力中国工程机械装备及配套产品开拓国际市场。


\begin{table} [ht] \footnotesize
	\newcommand{\tabincell}[2]{\begin{tabular}{@{}#1@{}}#2\end{tabular}}  %导言区
	\centering
	\captionsetup{labelformat=default,labelsep=period}
	\caption{中国工程类产品出口情况}\label{table3}
	\begin{tabular}{lcccc}
		\toprule
		\vspace{0.2em}
		&&  \tabincell{c}{中国工程类产品出口额占总 \\ 出口额的比重(单位:\%)}     & \tabincell{c}{工程类装备出口金额 \\ (单位:亿美元)} & \tabincell{c}{中国工程类产品出口占世界 \\ 比重(单位:\%)} \\
		\cline{1-5} 
		&2005&	44.76&	3410.41 &	11.64 \\
		&2006&	45.50&	4408.29 &	13.57 \\
		&2007&	46.36&	5656.36 &	14.71 \\
		&2008&	46.07&	6591.73 &	11.50 \\
		&2009&	47.51&	5708.90 &	13.09 \\
		&2010&	46.76&	7377.13 &	14.14 \\
		&2011&	44.81&	8507.17 &	14.26 \\
		&2012&	44.87&	9193.77 &	15.46 \\
		&2013&	45.35&	10017.89 &	16.14 \\
		&2014&	44.08&	10323.95 &	15.45 \\
		&2015&	44.83&	10191.24 &	15.69 \\
		&2016&  45.23&	9488.32 &	14.21 \\
		&2017&	45.87&	10382.97 &	13.87 \\
		&2018&	46.50&	11599.36 &	13.82 \\
		&2019&	46.33&	11575.28& 	14.13 \\
		&2020&	-&	-&	17.05 \\
		
		\bottomrule
	\end{tabular}
	\begin{minipage}{14cm}%
		\scriptsize\vspace{0.5em}
		注:数据来自UN Comtrade数据。
	\end{minipage}%
\end{table}


\subsection{典型案例分析}

中石化国际石油工程公司在2009年成功获得中国在科威特的第一个石油钻井项目,一举打破了欧美在科威特钻井市场的垄断,并在随后8年左右成为科威特最大的钻井承包商。在此过程中,中石化国际石油工程公司极力帮助中国石油钻井装备和产品进入科威特,中方相关生产企业也借力逐步占据科威特市场。本文通过该典型案例,探究对外承包工程项目对相关工程装备和产品出口的促进作用。

	\vspace{0.5em}  %段前0.5行
	
科威特位于西亚地区,国土面积仅相当于北京市大小,但石油储量占世界的10\%左右。2008年国际金融危机爆发,石油工程市场急剧萎缩。此时,中石化国际石油工程公司员工前往科威特开拓市场。当时,科威特石油钻井市场被欧美和本土公司完全占据。经过详细考察与缜密测算,最终中石化国际石油工程公司在2009年4月的投标竞争中以0.2\%的微弱报价优势拿下价值8.6亿美元15部钻机的钻井合同,由此一举打破欧美垄断。
起初,本地业主要求钻井承包商配备欧美装备,其对顶驱、封井器等相关装备的要求十分严格。在此情况下, 中石化国际石油工程公司通过以下两种方式极力推动科威特石油钻井装备产品国产化:(1)联系国内相关装备生产厂商进行合作,帮助厂商对接国际质量控制体系。(2)在科威特积极宣传“中国制造”装备,并与科威特国家石油公司管理层进行技术交流。在中石化国际石油工程公司的努力下,科威特国家石油公司同意以“试用”的名义引入封井器等中国装备。在此基础上,凭借质优价廉的竞争优势,中国装备及配套产品逐步得到科威特方面的认可,并迅速占领相关市场。

	\vspace{0.5em}  %段前0.5行

本文按照海关HS6位码筛选出石油钻井相关产品。图\ref{fig1}(A)反映了中国对科威特出口的所有HS6层面与石油钻井相关的产品总额。如图\ref{fig1}所示,在2009年中国第一个石油钻井项目承包之前,中国钻井类装备和产品在科威特的市场份额一直维持在5\%左右。但是在第一个项目承包之后,中国钻井相关产品出口骤增。2010年,中国市场份额就迅猛上升至17\%左右,并在2019年接近25\%。图\ref{fig1}的其他三幅图分别反映了石油钻井的三类主要产品出口情况,包括钢铁制品(HS2为73)、锅炉器件(HS2为84)和电气装备(HS2为85)。如图\ref{fig1}所示,这三类产品也在2009年之后呈现快速增长的趋势。

	\vspace{0.5em}  %段前0.5行

通过对中国在科威特的首个工程项目进行分析,本文能够得出如下三点基本结论。(1)中国在科威特首次获得石油钻井项目之后,中国对科威特相关装备及配套产品的出口骤增。(2)该案例也表明,本文的实证研究并不存在反向因果的内生性问题,即并不是相关装备产品出口增加促进了对外工程项目的承包。从科威特的案例分析来看,项目承包仅以0.2\%的微弱优势获胜,这说明竞标成功本身就具有相当大的偶然性。并且,在2009年首个项目承包之前,中国对科威特的钻井装备和产品的出口一直维持在低位。(3)科威特的案例阐释了项目承包推动相关产品出口的两条重要途径,包括帮助国内厂商对接国际质量控制体系,以及极力宣传中国产品,增进外方企业对中国制造的了解和信任。


\begin{figure}[htbp]
	\centering
	\includegraphics[width=9.5cm,height=8cm]{FIG/fig1.png}
	\captionsetup{labelformat=default,labelsep=period}
	\caption{中国在科威特石油钻井装备及配套产品的市场份额}\label{fig1}
\end{figure} 





\section{实证设计}
\subsection{模型的设定}

本文研究的问题是中国对外承包工程项目能否促进中国工程类相关装备和产品的出口。具体计量模型设定如下:
\begin{equation}\label{baseeq}
	lnexp_{it}=\beta_{0} + \beta_{1}OPC_{it} + X_{i}·f(t)+BR_{i}·Post_{2013}+\gamma_{i} +\gamma_{t}+\epsilon_{it}
\end{equation}
式\ref{baseeq}中,因变量$\ln{{exp}_{it}}$表示$t$年中国对$i$国工程类装备和产品出口额的对数。核心解释变量${OPC}_{it}$用于识别$t$年中国在i国承包工程项目的情况。本文对${OPC}_{it}$采用了两种定义方式,具体如下:(1)虚拟变量。${OPC}_{it}$=1表示$t$年中国在$i$国承包过工程项目,否则为零。(2)连续变量。${OPC}_{it}$为$t$年中国在$i$国承包工程项目合同的累计金额的对数值。第一种定义方式对应的是基准的双重差分模型,反映对外承包项目发生前后中国工程类装备和产品出口的平均变化。第二种定义方式则是在第一种方式的基础上,进一步考虑中国在同一个国家多次承包项目所产生的累计效应。$\epsilon_{it}$为随机扰动项,$\beta_1$为重点待估参数,倘若估计值为正,则表明对外承包工程能够有效促进中国对该国工程类装备和产品的出口。

\vspace{0.5em}  %段前0.5行

此外,本文引入国家固定效应$\gamma_i$以控制所有不随时间变动的国家特征变量,引入时间固定效应$\gamma_t$以控制各年份不同国家普遍面临的经济冲击。除上述固定效应以外,回归模型还引入初始期国家特征变量与时间变量的交互项$X_i·f(t)$,以控制不同特征的国家所特有的时间趋势。具体地,中国对外投资目的国的选择很可能会受到伙伴国经济发展水平或者经济规模的影响,同时中国对不同收入水平或者不同经济规模的国家工程类产品出口也可能存在不同的时间趋势。鉴于此,$X_i$包含初始期$i$国总体GDP规模、人均GDP水平以及初始期各国对中国工程类产品的依赖程度${Share}_i$,其中,我们利用世界各国在初始期从中国进口的工程类产品金额除以该国从世界进口的工程类产品总金额度量依赖程度${Share}_i$;$f(t)$为时间趋势函数,为了反映非线性的时间趋势,$f(t)$包含时间的一次项、二次项和三次项。引入$X_i·f(t)$交互项能够缓解因内生性问题导致的估计偏误。

\vspace{0.5em}  %段前0.5行

更进一步,由于工程类投资和贸易是“一带一路”倡议的重要内容,所以“一带一路”倡议会同时影响中国工程类项目的承包以及工程类产品的出口。因此倘若不对“一带一路”倡议的影响加以控制,那么本文将高估核心解释变量${OPC}_{it}$的系数。基于此,本文在模型\ref{baseeq}中引入${BR}_i·{Post}_{2013}$用以控制“一带一路”倡议的影响。其中,${BR}_i$为虚拟变量,用以识别国家$i$是否为“一带一路”沿线国家,如果国家$i$属于“一带一路”沿线国家,则${BR}_i=1$,否则${BR}_i=0$;${Post}_{2013}$用以识别“一带一路”提出的时间,对于2013年以后的年份,${Post}_{2013}=1$,否则${Post}_{2013}=0$。


\subsection{数据说明}
首先,本文依照《商品名称及编码协调制度的国际公约》HS2位码对工程类装备及产品进行界定,将与工程建设相关的产品及装备定义为工程类产品,具体而言,工程类产品包括以下四大类:(1)钢铁制品(HS2为73),如叶轮、油杯、弹簧、螺钉等;(2)机械器具(HS2为84),如燃油泵、继电器、轴承等;(3)电气装备(HS2为85),如电机、镇流器、控制电缆等;(4)精密仪器(HS2为90),例如钻机监视系统、出口排量传感器、蒸发温度表等。工程类装备及产品的出口额数据来源于联合国商品贸易统计数据库(UN Comtrade),该数据库涵盖了中国自1996年以来向各伙伴国出口工程类装备或产品的贸易额。本文选取2005-2019年中国对伙伴国工程类产品的出口额的对数值作为被解释变量,选取中国与世界在2005年向各国出口工程类产品的贸易额,利用二者之比衡量期初各国对中国工程类产品的依赖度。此外,本文为进行安慰剂检验,选取了同时期中国对伙伴国非工程类产品的出口额。

\vspace{0.5em}  %段前0.5行

其次,中国对外承包工程情况由笔者根据美国企业研究所(American Enterprise Institute)2020年秋季披露的《中国全球投资跟踪(China Global Investment Tracker)》数据整理得来,其包含了2005-2019年中国企业对外投资及承包工程的相关数据,涉及中国企业名称、伙伴国名称、项目类型、合同发生时间、合同金额等关键数据。其中本文最关注的是中国在伙伴国承包工程项目合同首次发生的时间以及中国对每一伙伴国所签署工程项目合同的累计金额。最后,控制变量中各国人均GDP和GDP总量的数据来自“佩恩表”9.0版本(Penn World Table Ver9.0)。本文选取2005年各国基于链式的购买力平价理论(PPP)计算的产出层面实际GDP作为“期初国家总体GDP”控制变量;选取2005年各国基于链式的购买力平价理论(PPP)计算的人均GDP作为“期初国家人均GDP”控制变量。上述变量的数据描述性统计见表4。




\begin{table} [ht] \footnotesize
	\centering
	\captionsetup{labelformat=default,labelsep=period}
	\caption{描述性统计}\label{table4}
	\begin{tabular}{clllll}
		\toprule		
		变量 & 观测值 & 均值 & 最小值 & 最大值 & 标准差 \\ 
		\cline{1-6} 
		工程类装备及产品出口额对数 & 1888 & 13.36 & 6.570 & 19.43 & 2.319 \\ 
		非工程类装备及产品出口额对数 & 1888 & 13.52 & 6.685 & 19.17 & 2.266 \\ 
		承包项目虚拟变量 & 1888 & 0.737 & 0 & 1 & 0.440 \\ 
		承包合同金额对数 & 1888 & 3.256 & 0 & 10.39 & 3.671 \\ 
		工程类对外投资虚拟变量 & 1888 & 0.713 & 0 & 1 & 0.452 \\ 
		工程类对外投资金额对数 & 1888 & 2.907 & 0 & 11.25 & 3.717 \\ 
		邻国承包项目虚拟变量(方式一) & 1888 & 0.844 & 0 & 1 & 0.363 \\ 
		邻国承包项目虚拟变量(方式二) & 1888 & 0.921 & 0 & 1 & 0.271 \\ 
		“一带一路”沿线国家虚拟变量 & 1888 & 0.331 & 0 & 1 & 0.471 \\ 
		“一带一路”时间虚拟变量 & 1888 & 0.407 & 0 & 1 & 0.491 \\ 
		初始人均GDP的对数 & 1888 & 9.953 & 7.197 & 11.94 & 1.167 \\ 
		初始总计GDP的对数 & 1888 & 16.80 & 8.343 & 18.01 & 1.385 \\ 
		初始从中国进口占总进口百分比的对数 & 1888 & -3.085 & -10.372 & -0.610 & 1.251 \\ 
		\bottomrule
	\end{tabular}
\end{table}
	

\section{实证检验结果}
\subsection{平行趋势检验:事件分析法}

双重差分法适用的一个前提条件是控制组与实验组在冲击发生之前具有相同的时间趋势。为更好地反映实验组和控制组之间的可比性,本文首先利用事件分析法进行平行趋势假设检验。具体的回归方程如下:

\vspace{0.5em}  %段前0.5行

\begin{equation}\label{evanteq}
	lnexp_{it}=\sum_{k=-8}^{10}\beta^{k}\eta_{it}^{k} + X_{i}·f(t)+BR_{i}·Post_{2013}+\gamma_{i} +\gamma_{t}+\epsilon_{it}
\end{equation}

需注意的是,中国对外承包工程项目时间t并非是同一年份,因此本文需对自然年份做标准化处理,标准化后的时间用k表示。具体地,中国在\ i国承包工程项目合同当年,k赋值为“0”,承包后1年k赋值为“+1”,承包前1年k赋值为“-1”,以此类推。

\vspace{0.5em}  %段前0.5行

在模型\ref{evanteq}中,$\eta_{it}^k$为经过标准化处理的时间虚拟变量,反映$i$国$t$年与该国发生项目承包年份的前后间隔。例如,$\eta_{it}^k=2$,表示$t$年为中国在$i$国承包工程项目之后的2年。其他变量均与模型\ref{baseeq}相同。本模型中需关注的是$\eta_{it}^k$的系数$\beta^k$,其反映了在第$k$期时,中国对实验组和控制组工程类产品出口额的平均差异。倘若$\beta^k$在$k<0$期间接近于0,则表示对外承包工程发生之前,中国对实验组和控制组国家的工程类产品出口没有显著差异,也即符合平行趋势假设。

\vspace{0.5em}  %段前0.5行

图\ref{fig2}汇报了以事件发生前一期$(k=-1)$为基期的平行趋势检验结果。图中圆圈表示由事件分析法得到的系数估计值${\hat{\beta}}^k$,圆圈两侧的实线表示95\%的置信区间。如图\ref{fig2}所示,对外承包工程发生的前8年,回归系数的绝对值均明显小于0.1,这表明实验组和控制组在事件发生的前8年内几乎不存在差异。相反,在$k>0$的期间,除了第1期,所有$\beta^k$的估计值均大于或十分接近0.1,这说明在对外承包工程发生之后,中国对相应国家的工程产品出口相对于其他国家显著增加。


\begin{figure}[htbp]
	\centering
	\includegraphics[width=9.5cm,height=8cm]{FIG/fig2.png}
	\captionsetup{labelformat=default,labelsep=period}
	\caption{平行趋势检验}\label{fig2}
\end{figure} 


在项目承包发生之后,参与项目的中方企业一方面向中国企业介绍对方国家的相关标准和要求,另一方面又向对方国家推荐中方产品,帮助中方与外方企业建立联系。在此过程中,中国相关企业凭借自身价格优势和质量保证,开始逐渐在竞争中取得优势地位。正因如此,如图\ref{fig2}所示,对外承包工程的出口促进效应在第一次项目承包之后持续存在。换言之,由于中方装备和产品的竞争优势,一旦具体工程项目增加了企业对中方装备和产品的使用和了解,对方就会继续保持对中方装备和产品的需求。


\subsection{基准估计结果}
表\ref{table5}汇报了基准估计结果。第(1)(2)(3)列均以“是否发生过对外承包工程”作为核心解释变量。这三列研究的是,相对于没有发生项目承包的国家,中国对发生过项目承包的国家在项目承包前后的出口变化。其中,第(1)列包含了国家固定效应和时间固定效应,第(2)列进一步加入初始期国家特征与时间趋势的交互项,第(3)列在模型中控制了“一带一路”沿线国家虚拟变量${BR}_i$与倡议提出时间的虚拟变量的交互项,以此控制“一带一路”倡议的影响。

\vspace{0.5em}  %段前0.5行

回归结果如表\ref{table5}所示。如表\ref{table5}所示,第(1)列中$OPC$的回归系数为0.205。在控制了初始期国家特征与时间趋势的交互项之后,第(2)列中$OPC$的回归系数为0.164。在进一步控制“一带一路”倡议的影响后,第(3)列中$OPC$的回归系数为0.160。并且,三列的估计系数均在1\%的水平上显著。由此表明,承包伙伴国的工程项目有助于促进中国对该国的工程类产品出口。从经济意义角度上看,第(3)列结果表明,相对于其他国家,对外承包工程项目能够导致中国对该国的工程装备和产品出口平均增加约16\%。

\vspace{0.5em}  %段前0.5行

为了进一步反映对外承包工程项目的累积效应,本文将第(4)(5)(6)列的核心解释变量替换为截至当年中国与伙伴国项目承包合同的累积金额,其他设定分别与第(1)(2)(3)列相同。如表\ref{table5}所示,OPC的系数均在1\%的水平上显著为正。根据第(6)列结果可知,承包工程累计合同金额提高1\%,中国工程类装备和产品出口额将增加约0.025\%。如表\ref{table1}第二列所示,中国从2005年左右开始对外项目承包的合同金额持续快速增长,0.025\%的估计弹性表明对外承包工程项目对中国工程类产品的总体出口产生了巨大的推动作用。



\begin{table} [ht] \footnotesize
	\newcommand{\tabincell}[2]{\begin{tabular}{@{}#1@{}}#2\end{tabular}}  %导言区
	\centering
	\captionsetup{labelformat=default,labelsep=period}
	\caption{基准回归结果}\label{table5}
	\begin{tabular}{lccccccc}
	\toprule
	&(1)&(2)&(3)&(4)&(5)&(6)&\\
	关键解释变量	&\tabincell{c}{是否发生 \\ 过对外 \\ 工程承包}&	\tabincell{c}{是否发生 \\ 过对外 \\ 工程承包}&	\tabincell{c}{是否发生 \\ 过对外 \\ 工程承包}&	\tabincell{c}{承包项目 \\ 累计金额 \\ 对数值}	&\tabincell{c}{承包项目 \\ 累计金额 \\ 对数值}	&\tabincell{c}{承包项目 \\ 累计金额 \\ 对数值} \\	
		\cline{1-7} \\
		\vspace{0.05em}  %段前0.5行
	$OPC$ &0.205\sym{***} & 0.164\sym{***}&0.160\sym{***}&0.033\sym{***}&0.026\sym{***}&0.025\sym{***} \\
		  &(0.056)   	  &(0.057)        &(0.055)       &(0.009)       &(0.009)       &(0.008)  \\
	$BR_{i}·Post_{2013}$ &               &               &0.067         &              &              &0.061 \\
	      &          	  &               &(0.068)       &              &              &(0.068)  \\	  
	$X_{i}·f(t)$  & &控制 &控制 & &控制 &控制 \\
	国家固定效应  & 控制&控制 &控制 &控制 &控制 &控制 \\
	实践固定效应  & 控制&控制 &控制 &控制 &控制 &控制 \\
	观测值 &1888&1888&1888&1888&1888&1888\\
	$adj.R^{2}$ &0.662&	0.686&	0.687&	0.662&	0.686&	0.687\\
	\bottomrule
	\end{tabular}
	\begin{minipage}{14cm}%
	\scriptsize\vspace{0.5em}
	注:*、**和***分别表示10\%、5\%和1\%的显著水平,括号中的数值为在国家层面聚类调整的稳健标准误。下表同。
	\end{minipage}%
\end{table}

\subsection{异质性检验}
基础设施建设的国际合作是“一带一路”倡议的重要内容。在“一带一路”倡议的影响下,中方企业可能更容易通过项目承包帮助中国企业进入和占据对方相关产品市场。由此,本文进一步检验“一带一路”倡议是否能放大工程项目承包的贸易促进效应。具体的,本文通过如下三重差分回归模型进行验证:


\begin{equation}\label{heteq}
	\begin{aligned}
	lnexp_{it}=\beta_0+\beta_1·{OPC}_{it}\times{BR}_i\times{POST}_{2013}+\\ 
	\beta_2·{BR}_i\times{POST}_{2013}+\beta_3·{OPC}_{it}\times{BR}_i+\\ 
	\beta_4·{OPC}_{it}\times{POST}_{2013}+\beta_5·{OPC}_{it}+X_i·f(t)+\gamma_i+\gamma_t+\epsilon_it 
	\end{aligned}
\end{equation}


在式\ref{heteq}的回归中,本文重点关注$\beta_5$和$\beta_1$的系数。其中,$\beta_5$反映了对外承包工程项目的平均效应,$\beta_1$则进一步反映对于“一带一路”沿线国家的异质性影响在倡议提出前后的不同。表6报告了上述三重差分的回归结果。同样地,第(1)列中$OPC$为是否签署过承包工程项目合同的哑变量,第(2)列OPC为累计项目合同金额。

\vspace{0.5em}  %段前0.5行

如表\ref{table6}第(1)列所示,$\beta_5$的系数仍然显著为正。这表明平均而言,对外承包工程项目对于相关产品出口具有促进作用。此外,${OPC}_{it}\times{BR}_i\times{POST}_{2013}$的估计系数为0.186,且在10\%的水平上显著。这表明中国在“一带一路”沿线国家承包工程项目的出口促进效应较非沿线国家而言在2013年之后扩大了18.6\%。以累计合同金额为核心解释变量,表\ref{table6}第(2)列$\beta_5$的系数仍然显著为正。同时,尽管$\beta_1$的系数在统计意义上不显著,但是三项交互项的系数仍然为正。这同样表明“一带一路”倡议对海外项目承包的出口促进效应具有放大的作用。



\begin{table} [ht] \footnotesize
	\newcommand{\tabincell}[2]{\begin{tabular}{@{}#1@{}}#2\end{tabular}}  %导言区
	\centering
	\captionsetup{labelformat=default,labelsep=period}
	\caption{异质性影响:是否为“一带一路”沿线国}\label{table6}
	\begin{tabular}{lcc}
		\toprule
		&(1)&(2)\\
		关键解释变量	&\tabincell{c}{是否发生过 \\ 对外工程承包}&\tabincell{c}{承包项目累计 \\ 金额对数值} \\	
		\cline{1-3} \\
		\vspace{0.05em}  %段前0.5行
		$OPC$ &0.177\sym{***} & 0.023\sym{**} \\
				&(0.062)   	  &(0.009)     \\
		${OPC}_{it}\times{BR}_i\times{POST}_{2013}$ & 0.186\sym{*}     & 0.018       \\
		                                    &    (0.109)     	  &     (0.013)                 \\	  
		$X_{i}·f(t)$  & 控制&控制   \\
		国家固定效应  & 控制&控制  \\
		实践固定效应  & 控制&控制   \\
		观测值 &1888&1888\\
		$adj.R^{2}$ &0.687&	0.688\\
		\bottomrule
	\end{tabular}
\end{table}

考虑到各国的初始经济发展水平和经济规模相异,以及各国初始期对中国出口产品的依赖度也各不相同,本文在回归模型中引入交互项,进一步检验对外承包工程项目对出口的促进效应在初始期国家人均收入水平(gdp)、经济规模(GDP)以及对中国出口依赖度(share)等方面的异质性分析。

\vspace{0.5em}  %段前0.5行


如表\ref{table7}所示,无论是以“是否发生过对外工程承包”为核心解释变量,还是以“签署合同的累计金额”为核心解释变量,OPC的估计系数都为正。同时,三个交互项的系数均为负,其中与初期国家人均GDP的交互项和与初期国家总GDP的交互项均不显著,与初期中国出口依赖度的交互项分别在5\%或者1\%的水平上显著。值得注意的是,由于对中国出口依赖度的定义,该变量始终是介于0和1之间的连续变量。所以,对中国出口依赖度的对数均小于0。正因如此,表7中交互项$OPC·share$的估计系数显著为负,恰恰表明对外项目承包能够有效促进工程类产品的出口。具体而言,表7的异质性分析表明,对于收入水平较高、经济规模较大的国家,对外承包工程项目的促进效应与收入水平较低、经济规模较小的国家并无显著差异。此外,对外承包项目的出口促进效应对于初始期从中国进口比重较小(即初始期对中国出口依赖度较小)的国家影响更大。对外承包工程项目之所以能够推动相关产品和装备的出口,一个重要的中间渠道就是中方企业在项目实施过程中主动使用中方装备和产品,进而帮助国外企业了解和接受中方装备和产品。对于那些初始期从中国进口工程装备和产品较少的国家而言,中方承包工程项目所产生的信息学习和信息传递的作用更大,因而更能促进中国对这些国家的工程产品出口。

\begin{table} [ht] \footnotesize
	\newcommand{\tabincell}[2]{\begin{tabular}{@{}#1@{}}#2\end{tabular}}  %导言区
	\centering
	\captionsetup{labelformat=default,labelsep=period}
	\caption{异质性分析:基于标的国特征的研究}\label{table7}
	\begin{tabular}{lcc}
		\toprule
		&(1)&(2)\\
		关键解释变量	&是否发生过对外工程承包&承包项目累计金额对数值 \\	
		\cline{1-3} \\
		\vspace{0.05em}  %段前0.5行
		$OPC$ &0.645 & 0.075 \\
		&(1.010)   	  &(0.145)     \\
		${OPC}_{it}\times GDP$ & -0.049     & -0.005      \\
		&    (0.048)     	  &     (0.007)                 \\	  
		${OPC}_{it}\times gdp$ & -0.009     & -0.002       \\
		&    (0.048)     	  &     (0.007)                 \\	  
		${OPC}_{it}\times share$ & -0.146\sym{**}     & -0.052\sym{***}        \\
		&    (0.057)     	  &     (0.013)                 \\	  
		$X_{i}·f(t)$  & 控制&控制   \\
		国家固定效应  & 控制&控制  \\
		实践固定效应  & 控制&控制   \\
		观测值 &1888&1888\\
		$adj.R^{2}$ &0.675&	0.683\\
		\bottomrule
	\end{tabular}
\end{table}



\subsection{安慰剂检验}

在21世纪初“走出去”战略、2013年“一带一路”倡议的背景下,中国对外承包工程项目数量与日俱增,各类商品出口额均呈现快速增长态势。因此本文存在的一个潜在问题是,对外承包工程项目和工程类产品出口额的增加都是由伙伴国层面外生的环境与政策因素导致的,而并非如本文所述对外承包工程项目促进了工程类装备和产品的出口。为排除此担忧,表\ref{table8}进行了相应的安慰剂检验。如果之前的回归结果为伪回归,那么以非工程类产品作为被解释变量对海外工程项目承包的变量OPC进行回归,OPC的系数依然显著为正。相反,如果其估计系数明显小于以工程类产品为被解释变量得到的估计系数或者不显著,则说明对外承包工程对工程类产品出口的影响显著大于非工程类产品的影响,也即对外承包工程的出口促进效应并非来自其他国家层面的因素影响。

\vspace{0.5em}  %段前0.5行

如表\ref{table8}所示,对外工程项目承包对非工程类产品的出口不存在显著影响。同时,从系数大小看,表\ref{table8}第(1)(2)列的回归系数较表\ref{table3}第(1)(2)列分别小50\%和36\%。这表明对外承包工程对工程类产品出口的促进作用并非来自其他国家层面没有控制的冲击所致。

\begin{table} [ht] \footnotesize
	\newcommand{\tabincell}[2]{\begin{tabular}{@{}#1@{}}#2\end{tabular}}  %导言区
	\centering
	\captionsetup{labelformat=default,labelsep=period}
	\caption{异质性分析:基于标的国特征的研究}\label{table8}
	\begin{tabular}{lcc}
		\toprule
		&(1)&(2)\\
		关键解释变量	&是否发生过对外工程承包&承包项目累计金额对数值 \\	
		\cline{1-3} \\
		\vspace{0.05em}  %段前0.5行
		$OPC$ &0.088\sym{***} & 0.016\sym{**} \\
		&(0.059)   	  &(0.010)     \\
		$X_{i}·f(t)$  & 控制&控制   \\
		国家固定效应  & 控制&控制  \\
		实践固定效应  & 控制&控制   \\
		观测值 &1888&1888\\
		$adj.R^{2}$ &0.729&	0.730\\
		\bottomrule
	\end{tabular}
\end{table}

\subsection{进一步分析}
\subsubsection{对外直接投资的影响}
对外直接投资和对外承包工程是企业“走出去”的两种形式,大量文献也验证了对外直接投资的出口效应(张春萍2012;毛其淋等,2014;蒋冠宏和蒋殿春,2014;刘海云等,2016)\upcite{ref13}\upcite{ref14}\upcite{ref15}\upcite{ref16}。考虑到对外直接投资和对外承包工程可能存在正相关关系,且两者均可能对中国工程类产品的出口产生正向影响,本节将工程项目承包与工程直接投资同时纳入回归模型,以排除对外直接投资的影响。
表\ref{table9}汇报了相关回归结果。将对外工程项目承包与工程类对外直接投资分别或同时放入回归模型,不论是虚拟变量形式还是连续变量形式,工程项目承包(OPC)的估计系数均在1\%的水平上显著为正,且系数大小与表\ref{table5}的基准回归结果十分接近。这表明在排除了对外直接投资的影响之后,对外承包工程的出口效应依然稳健。

\vspace{0.5em}  %段前0.5行

相反,工程对外直接投资(OFDI)的估计系数均不显著,且系数大小明显小于OPC的估计系数。这可能有两方面原因。一方面,对外工程项目承包的过程中,具体项目实施由中方企业完成。在项目实施的过程中,中国企业会积极主动地使用中国制造装备及配套产品,因而派生出巨大需求。相比之下,部分对外直接投资可能涉及股权投资和收购,具体的生产活动仍由外方企业完成,因此使用中国产品的激励更小。另一方面,如表2所示,对外承包工程中国企所占比重远远大于对外直接投资。由于国有企业承担更多的社会责任,外溢效应更强,所以它对其他企业的出口带动作用更加明显。 

\begin{table} [ht] \footnotesize
	\newcommand{\tabincell}[2]{\begin{tabular}{@{}#1@{}}#2\end{tabular}}  %导言区
	\centering
	\captionsetup{labelformat=default,labelsep=period}
	\caption{两种“走出去”方式的比较}\label{table9}
	\begin{tabular}{lcccccc}
		\toprule
		&(1)&(2)&(3)&(4)&(5)&(6)\\
		关键解释变量	&\multicolumn{3}{c}{承包或投资合同是否发生} &\multicolumn{3}{c}{承包项目累计金额对数值}  \\
		\cline{1-7} \\
		\vspace{0.05em}  %段前0.5行
		$OPC$ &0.164\sym{***} & &0.163\sym{***} &0.026\sym{***}&&0.025\sym{***} \\
			  &(0.057)   	  &  &(0.057)       &(0.009)       &    &(0.009)  \\
		$OFDI$ & & 0.035		&0.018			&&0.010       &0.017 \\
			   & &(0.044)        &(0.043)       &&(0.007)       &(0.043)  \\
		$X_{i}·f(t)$  & &控制 &控制 & &控制 &控制 \\
		国家固定效应  & 控制&控制 &控制 &控制 &控制 &控制 \\
		实践固定效应  & 控制&控制 &控制 &控制 &控制 &控制 \\
		观测值 &1888&1888&1888&1888&1888&1888\\
		$adj.R^{2}$ &0.686&	0.681&	0.686&	0.686&	0.682&	0.686\\
		\bottomrule
	\end{tabular}
\end{table}

\subsubsection{“邻国”效应检验}
我们进一步探讨工程项目承包的影响是否在国家之间延申,即中国在一国的邻国承包工程项目能否增加中国对于该国的设备产品出口。具体地,我们将基准回归中“中国是否在该国承包过工程项目”(即本国OPC)的变量替换为“中国是否在该国的邻国承包过工程项目”(即邻国OPC),或者在基准回归中加入该变量。变量邻国OPC等于1,即表示中国在该国的邻国承包过工程项目。关于“邻国”的判断,我们采用了两种定义方式,以保证结果的稳健性:第一,按照两国是否接壤来界定是否为邻国。若j与i接壤,则界定j为i的“邻国”。第二,按照国家间距离是否足够小(即小于所有国家两两之间地理距离的5\%分位数。若j与i的距离小于5\%分位数,则界定j为i的“邻国”。具体的回归结果如表\ref{table10}所示。其中,第(1)(2)列采用方式一定义 “邻国”,第(3)(4)列采用方式二定义“邻国”。如表\ref{table10}所示,变量邻国OPC的系数十分接近于0,但是本国OPC的系数与表\ref{table5}基准回归第(2)列几乎相同。由此表明,中国在一国的邻国承包工程项目对该国几乎没有影响。

\vspace{0.5em}  %段前0.5行

实际上,表\ref{table10}的回归结果进一步增强了本文回归的可靠性。本质上讲,邻国工程承包如果对本国存在影响,那么就是双重差分中处理组受到的冲击对控制组存在影响,即处理组对于控制组存在溢出效应。溢出效应的存在又会导致双重差分的识别出现偏误。因此,表\ref{table10}的结果恰好可以消除这一疑虑。

\begin{table} [ht] \footnotesize
	\newcommand{\tabincell}[2]{\begin{tabular}{@{}#1@{}}#2\end{tabular}}  %导言区
	\centering
	\captionsetup{labelformat=default,labelsep=period}
	\caption{邻国OPC对中国工程机械设备出口的影响}\label{table10}
	\begin{tabular}{lcccc}
		\toprule
		邻国定义方式 &\multicolumn{2}{c}{方式一}&\multicolumn{2}{c}{方式二} \\
		\cline{1-5}
		&(1)&(2)&(3)&(4)\\
		关键解释变量	&\tabincell{c}{是否发生过 \\ 对外工程承包}&	\tabincell{c}{是否发生过 \\ 对外工程承包}&	\tabincell{c}{是否发生过 \\ 对外工程承包}&	\tabincell{c}{是否发生过 \\ 对外工程承包}\\	
		\cline{1-5} \\
		\vspace{0.05em}  %段前0.5行
		邻国$OPC$ &0.012          & 0.014         &0.010         &-0.037   \\
				 &(0.035)   	  &(0.035)        &(0.046)       &(0.043)       \\
		本国$OPC$ &   & 0.151\sym{***}&&0.149\sym{***}\\
		          &  &(0.031)        &&(0.032)    \\
		$X_{i}·f(t)$  &控制 &控制 &控制 &控制  \\
		国家固定效应  & 控制&控制 &控制 &控制  \\
		实践固定效应  & 控制&控制 &控制 &控制  \\
		观测值 &1888&1888&1888&1888\\
		$adj.R^{2}$ &0.67&	0.68&	0.67&0.68\\
		\bottomrule
	\end{tabular}
	\begin{minipage}{14cm}%
	\scriptsize\vspace{0.5em}
	说明: 第(1)(2)列采用方式一定义 “邻国”,第(3)(4)列采用方式二定义“邻国”。
	\end{minipage}%
\end{table}



\section{结论与建议}
机械装备是中国出口的重要组成部分,目前约占中国制造业出口总额的近45\%。从产品属性分析,工程机械装备具有如下典型特征:单价高,购买数量少,购买次数少。由此,从购买者的角度出发,缺乏激励尝试购买和使用来自新生产厂家的机械装备。在最初欧美厂商几乎垄断工程机械国际市场的背景下,中国国产装备开拓国际市场面临极大的挑战。本文的研究表明,中国对外承包工程的迅速发展为中国国产装备打破欧美垄断,抢占国际市场发挥了巨大作用。

\vspace{0.5em}  %段前0.5行

对外承包工程是中国企业“走出去”的重要形式。尤其是在“一带一路”倡议提出之后,沿线国家的工程承包更是得到进一步发展。参与对外承包工程的中国企业不仅主动使用国产工程机械装备,也积极发挥中介桥梁作用,帮助中国机械装备生产企业进入标的国市场。通过案例分析,本文发现自2009年中石化承包中国在科威特的首个石油钻井项目之后,中国对于科威特出口的钻井装备和配套产品激增,且在后续年份保持了持续稳定的增长。此外,案例分析进一步明确了中间机制,即中方企业在承包海外工程项目时积极推介本国工程产品并帮过国内企业协调产品标准,从而带动相关产品或装备出口。在此基础上,本文进一步采用双重差分方法验证了对外承包工程项目对中国装备制造业出口的促进作用,并进一步发现在“一带一路”倡议提出之后对于沿线国家出口的促进作用有所增强,对于初始期较少使用中国装备制造产品的国家出口促进作用越强。此外,本文对比了两种“走出去”方式——对外直接投资和海外工程承包,发现海外工程承包更能够促进国产装备进入和占据标的国市场。最后,本文也进一步考察了邻国效应,即中国在一国承包工程项目能否促进中国对该国的邻国装备出口。结果表明,邻国效应较弱,这反过来增强了本文双重差分的可靠性。

\vspace{0.5em}  %段前0.5行

基于上述研究结果,本文的政策建议如下。首先,政府应该继续鼓励中方企业参与对外承包工程,完善相关配套政策和公共服务。目前中国已进入接近90个国家的海外工程承包市场,政府可以适当鼓励企业继续开拓海外工程项目市场,尤其是对中国装备接受度不高的国家,从而继续发挥企业 “走出去”对中国装备出口的带动效应。其次,增强国内机械装备生产企业与参与海外工程项目承包的中方企业之间的沟通和联系,从而更好地发挥承包企业中间桥梁的作用。并充分利用“一带一路”倡议带来的契机,一方面通过与沿线国家基建工程合作促进中国装备对沿线国家的出口,另一方面又通过中国提供的优质装备促进沿线国家工程建设,最终实现双赢局面。最后,由于参与海外工程承包的企业绝大部分属于国有企业,本文的研究指出了国有企业“走出去”对于其他企业出口的积极作用。政府应该继续鼓励和引导国有企业在出口市场发挥平台溢出效应。






%% 用\bibliography对文献进行引用
\bibliographystyle{plain}
\bibliography{LR}

\begin{thebibliography}{99}  
		\bibitem{ref1}	曾剑宇,蒋骄亮,何凡.东道国国家风险与我国对外承包工程——基于跨国面板数据的实证研究[J].国际商务(对外经济贸易大学学报),2017(6):6-18.
		\bibitem{ref2}	何凡,曾剑宇.我国对外承包工程受双边关系影响吗?——基于“一带一路”沿线主要国家的研究[J].国际商务研究,2018, (6):57-66.
		\bibitem{ref3}	蒋骄亮,何凡,曾剑宇.人民币汇率与对外承包工程——基于跨国面板数据的实证研究[J].投资研究,2017,(1):38-50.
		\bibitem{ref4}	吕荣杰,张冰冰,张义明.我国对外承包工程与出口贸易关系研究——基于VAR模型的脉冲响应与方差分解[J].国际商务(对外经济贸易大学学报),2018(04):46-57.
		\bibitem{ref5}	曾剑宇,何凡,蒋骄亮.我国对外承包工程推动东道国产业结构升级了吗——基于跨国面板数据的实证研究[J].国际经贸探索,2017,(8):38-56.
		\bibitem{ref6}	李者聪.对外承包工程空间溢出性与东道国经济增长——“一带一路”沿线非洲国家的实证分析[J].国际商务研究,2021,(6):35-46.
		\bibitem{ref7}	徐俊,李金叶.中国对外承包工程提升东道国基础设施质量吗?:基于85个国家跨国面板数据的实证研究[J].世界经济研究,2020(5):111-122.
		\bibitem{ref8}	杨忻,刘芳,张国清.对外承包工程对中国经济的影响及政策研究[J].国际贸易,2005(6):17-18.
		\bibitem{ref9}	蔡阔,邵燕敏,何菊香,汪寿阳.对外承包工程对中国对外直接投资的影响——基于分国别面板数据的实证研究[J].管理评论,2013,(9):21-28.
		\bibitem{ref10}	覃伟芳,陈红蕾.对外承包工程“走出去”与工业企业出口扩张[J].国际商务(对外经济贸易大学学报),2018(2):53-62.
		\bibitem{ref11}	喻春娇.中国对外承包工程促进了钢铁产品出口吗?——基于“一带一路”沿线国家基础设施质量中介效应的研究[J].湖北大学学报(哲学社会科学版),2021,(2):144-153.
		\bibitem{ref12}	洪俊杰,詹迁羽.“一带一路”设施联通是否对企业出口有拉动作用——基于贸易成本的中介效应分析[J].国际贸易问题,2021(9):138-156.
		\bibitem{ref13}	张春萍.中国对外直接投资的贸易效应研究[J].数量经济技术经济研究,2012,(6):74-85.
		\bibitem{ref14}	毛其淋,许家云.中国对外直接投资促进抑或抑制了企业出口?[J].数量经济技术经济研究,2014,(9):3-21.
		\bibitem{ref15}	蒋冠宏,蒋殿春.中国企业对外直接投资的“出口效应”[J].经济研究,2014,(5):160-173.
		\bibitem{ref16}	刘海云,毛海欧.制造业OFDI对出口增加值的影响[J].中国工业经济,2016(7):91-108.
\end{thebibliography}

	
\end{document}
